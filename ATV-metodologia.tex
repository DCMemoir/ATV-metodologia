\documentclass[12pt,a4paper]{memoir}
\usepackage[utf8]{inputenc}
\usepackage[T1]{fontenc}
\usepackage[brazil]{babel}
\usepackage{graphicx}
\usepackage{textcmds}
\usepackage{xurl}
\usepackage[margin=2.5cm]{geometry}
\usepackage[left,modulo]{lineno}
\usepackage{lipsum}

%Definições Memoir
\OnehalfSpacing
\pagestyle{empty}
\setlength{\textwidth}{6.5in}
\setlength{\textheight}{9.in}
\setlength{\oddsidemargin}{0in}
\setlength{\headheight}{0in}

%Informações do cabeçalho.
\newcommand{\metod}{\noindent\large \textsc{\textbf{Metodologia -- 3\textsuperscript{o} ADS}}}

%Docente
\newcommand{\aapl}{\noindent\large \textsc{\textbf{André Padilha}}}

%Informações do discente e curso/turno.
\newcommand{\estudante}[1]{\noindent\normalsize\textbf{Estudante:} #1\\}


%Informações da atividade.
\newcommand{\atividade}[1]{
	\vspace{5mm}
	{\textsc{\textbf{Atividade:}}} {#1}
	\ \\
}

%Informações da data de entrega.
\newcommand{\dataentrega}[1]{\indent\normalsize\textbf{Data de entrega:} #1}

%Configurações do logo do IFPE no cabeçalho.
\newcommand{\logo}{
	\begin{minipage}{3cm}
		\begin{flushright}
			\includegraphics[scale=.15]{img/logo-ifpe.png}
		\end{flushright}
	\end{minipage}	
}








\begin{document}
%Preencha abaixo apenas os campos entre chaves para os comandos \disciplina, \docente, \nome, \matricula, \atividade e \dataentrega
%===========================================================
%Preencha abaixo apenas os campos entre chaves para os comandos \disciplina, \docente, \estudante, \atividade e \dataentrega
\begin{minipage}{12cm}
\metod\\
\aapl\\
\estudante{Digite seu nome aqui}\\ %Separe com ' ; ' (ponto e vírgula) quando for mais de um estudante.
\dataentrega{Digite a data de entrega.}

\end{minipage}
\logo\hrule
\atividade{Por exemplo: Exercício 1.}
\newline
%====================INSTRUÇÕES============================
%Comece seu texto abaixo da linha indicada com \begin{linenumbers}.
%Para um novo parágrafo, pressione "enter" 2x após o final do parágrafo que antecede.
%Tudo que você digitar deve estar entre os comandos \begin{linenumbers} e \end{linenumbers}. 
%A numeração de linhas é automática e aparece do lado esquerdo. A contagem é feita de 5 em 5.
%Alguns comandos básicos abaixo:
% Negrito = \textbf{negrito}; Itálico = \textit{italico} e sublinhado = \underline{sublinhado}
% Aspas duplas = \qq{aspas duplas}; Aspas simples = \q{aspas simples}
%Não digite nada após a linha "\end{document}"
%===========================================================
\begin{linenumbers}
\lipsum[3]


\end{linenumbers}
\end{document}

